\documentclass[a4paper,10pt]{article}
\usepackage{epsfig}
\usepackage{latexsym}
\usepackage{graphicx}
\usepackage{amsfonts}
\usepackage{amsmath}
\usepackage{xcolor}

%-----------------------------------------
% Pour accepter les lettres accentuees de clavier azerty
% sans les \'e (utile) pour tapper directement en azerty 
% et ou faire passer aspell -c --lang=fr bidon.tex
%\usepackage[latin1]{inputenc}
%------------------------------------------


%\topmargin=-3cm
\topmargin=-1cm
\oddsidemargin=-1cm
\evensidemargin=-1cm
\textwidth=17cm
%\textheight=27cm
\textheight=25cm
\raggedbottom
\sloppy

\definecolor{Blue}{rgb}{0.,0.,1.}
\definecolor{LightSkyBlue}{rgb}{0.691,0.827,1.}
\definecolor{Red}{rgb}{1.,0.,0.}
\definecolor{Green}{rgb}{0.,1.,0.}
\definecolor{Purple}{rgb}{0.5, 0., 0.5}
\definecolor{Try}{rgb}{0.15,0.,1}
\definecolor{Black}{rgb}{0., 0., 0.}

\title{NIKA: Notice for Observers}
\author{N.~Ponthieu\footnote{Nicolas.Ponthieu@obs.ujf-grenoble.fr}}
%Institut de Plan\'etologie et d'Astrophysique de Grenoble (IPAG),\\
% CNRS and Universit\'e de Grenoble,\\
%France}

\begin{document}
\maketitle

\abstract{This notice gathers instructions for observers using the NIKA
  instrument.}

\section{Getting Ready}

\begin{enumerate}
\item From the main control terminal, log on {\tt mrt-lx1.iram.es}. Let's call
  T1 this terminal that will be dedicated to PAKO.
\item From anothe terminal, log on {\tt mrt-lx1.iram.es}. Let's call T2 this
  terminal dedicated to IDL real time analaysis.
\item From another terminal, log on {\tt mrt-lx3.iram.es}. Let's call T3 this
  terminal, that will be dedicated to Xephem.
\end{enumerate}

\subsection{PaKo}
In T1:
\begin{enumerate}
\item type {\tt goPako}
\item type {\tt PakoDisplay}
\item type {\tt PakoNIKA}
\end{enumerate}

\subsection{Xephem}
In T3:
\begin{enumerate}
\item type {\tt ps xa |grep azElToXephem.py \&} to check if the script is
  already running.
\item If yes, then do nothing. If not, then type {\tt azElToXephem.py \&}
\item Once the script is running, type {\tt xephem \&}
\end{enumerate}

\subsection{IDL set up}
In T2, you will log in and work on SAMI, the computer dedicated to NIKA data
processing during observations:
\begin{enumerate}
\item type {\tt ssh\_sami}\footnote{This is an alias to {\tt ssh observer@150.214.224.22 -Y} with password {\tt nika30m}.}
\item type rt. This will put you in the ``Realtime'' directory.
\item type {\tt emacs \&}. This will allow you to edit the scripts
  relevant for real time data analysis.
\item type {\tt idl}
\end{enumerate}

\section{Analyzing data}

Shortly after a scan is done, the NIKA scientific data and the AntennaIMBfits
are written on SAMI and can be processed. There are two types of observation:
the ``science'' scans and the ``calibration'' scans. There is a specific routine
to analyze each type of scan.

\begin{itemize}
\item The science scans are meant
  to be optimally processed offline with taylored procedures. The real time
  software mentionned in this note only aims at giving a quick feedback
\item The calibration scans expect actions from the observer and interaction with
  PaKo.
\end{itemize}

\subsection{Calibration scans}

\subsubsection{Pointing}

\begin{enumerate}
\item In the emacs window, edit run\_pointing.pro and update the {\tt day} and
  {\tt scan\_num} parameters.
\item update the {\tt p2cor} and {\tt p7cor} parameters with the ``SET
  POINTING'' values displayed in the PAKO window.
\item Save the file
\item In the idl session in T2, type {\tt .r run\_pointing}
\item Several plot windows might be on top of eachother
\item Follow instructions returned by the code in T2. By default, follow
  instructions ``B (2mm) : (MAP) (for PAKO and pointing model)''.
\end{enumerate}

\subsubsection{Pointing\_liss}

\begin{enumerate}
\item In the emacs window, edit run\_pointing\_liss.pro and update the {\tt day}
  and {\tt scan\_num} parameters.
\item Save the file
\item In the idl session in T2, type {\tt .r run\_pointing\_liss}
\item Several plot windows might be on top of eachother
\item Follow instructions returned by the code in T2
\end{enumerate}

\subsubsection{Focus}

\begin{enumerate}
\item In the emacs window, edit run\_focus.pro and update the {\tt day} and {\tt
  scan\_num} parameters.
\item Check the value of focusz in the PAKO display window.
\item Do not change the {\tt fooffset} values unless you have modified the pako
  script {\tt focusp.pako}.
\item Save the file
\item In the idl session in T2, type {\tt .r run\_focus}
\item Several plot windows might be on top of eachother
\item Follow instructions returned by the code in T2
\end{enumerate}

\subsubsection{Focus\_liss}

\begin{enumerate}
\item In the emacs window, edit run\_focus\_liss.pro and update the {\tt day}
  and {\tt scan\_num} parameters. {\b Scan num should be the first of the five
    scans involved in this analysis}.
\item Save the file
\item In the idl session in T2, type {\tt .r run\_focus\_liss}
\item Several plot windows might be on top of eachother
\item Follow instructions returned by the code in T2
\end{enumerate}

\subsubsection{Skydip}

\begin{enumerate}
\item In the emacs window, edit run\_skydip.pro and update the {\tt day}
  and {\tt scan\_num} parameters.
\item Save the file
\item In the idl session in T2, type {\tt .r run\_skydip}
\item Several plot windows might be on top of eachother
\item Follow instructions returned by the code in T2
\end{enumerate}


\subsubsection{OTF\_geometry}

\begin{enumerate}
\item In the emacs window, edit run\_otf\_geometry.pro and update the {\tt day}
  and {\tt scan\_num} parameters.
\item Save the file
\item In the idl session in T2, type {\tt .r run\_otg\_geometry}
\item Several plot windows might be on top of eachother
\item Follow instructions returned by the code in T2
\end{enumerate}

\subsection{Science scans}

\subsubsection{Total power maps}

All maps (OTF or Lissajou) in total power mode can be reduced by the same script:

\begin{enumerate}
\item In the emacs window, edit run\_otf\_map.pro and update the {\tt scan\_num}
  and {\tt day} parameters.
\item If you're observing a point source, set {\tt diffuse = 0}. If you're
  observing diffuse emission set {\tt diffuse = 1}.
\item Save the file
\item In the idl session in T2, type {\tt .r run\_otf\_map}
\item Several plot windows might be on top of eachother
\end{enumerate}

\subsubsection{Polarization maps}

All maps (OTF or Lissajou) in Polarization modes can be reduced by the same script:
\begin{enumerate}
\item In the emacs window, edit run\_otf\_polar\_maps.pro and update the {\tt scan\_num}
  and {\tt day} parameters.
\item Save the file
\item In the idl session in T2, type {\tt .r run\_otf\_polar\_maps}
\item Several plot windows might be on top of eachother
\end{enumerate}


%----------------------------------------------------------------------------------------
%\begin{thebibliography}{}
%\end{thebibliography}

\end{document}
