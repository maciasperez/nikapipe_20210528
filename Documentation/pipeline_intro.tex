\documentclass[a4paper,10pt]{article}
\usepackage{epsfig}
\usepackage{latexsym}
\usepackage{graphicx}
\usepackage{amsfonts}
\usepackage{amsmath}
\usepackage{xcolor}

%-----------------------------------------
% Pour accepter les lettres accentuees de clavier azerty
% sans les \'e (utile) pour tapper directement en azerty 
% et ou faire passer aspell -c --lang=fr bidon.tex
%\usepackage[latin1]{inputenc}
%------------------------------------------


%\topmargin=-3cm
\topmargin=-1cm
\oddsidemargin=-1cm
\evensidemargin=-1cm
\textwidth=17cm
%\textheight=27cm
\textheight=25cm
\raggedbottom
\sloppy

\definecolor{Blue}{rgb}{0.,0.,1.}
\definecolor{LightSkyBlue}{rgb}{0.691,0.827,1.}
\definecolor{Red}{rgb}{1.,0.,0.}
\definecolor{Green}{rgb}{0.,1.,0.}
\definecolor{Purple}{rgb}{0.5, 0., 0.5}
\definecolor{Try}{rgb}{0.15,0.,1}
\definecolor{Black}{rgb}{0., 0., 0.}

\title{Overview of the NIKA(2) IDL pipeline}
\author{N. Ponthieu et al}

\begin{document}
\maketitle

\abstract{This note gives a big picture of how the IDL pipeline works. It
  complements the wiki, in particular the FAQ page.}

%% \begin{figure}
%% \begin{center}
%% \includegraphics[clip, angle=0, scale = 0.5]{beam_conventions.eps}
%% \caption{
%% Elliptical beam parameters definition. $\beta$ is the angle between
%% the South/North axis and the $x$ axis of the focal plane. The beam is
%% tilted w.r.t $x$ by an angle $\vartheta$. The detector is sensitive to a
%% direction of polarization $\psi$. The gaussian FWHM in the directions
%% $x$ and $y$ are resp. $2.35\sigma_x$ and $2.35\sigma_y$.}
%% \label{fig:beam_params}
%% \end{center}
%% \end{figure}


\section{Introduction}

The current IDL pipeline was built on an early version developped for
NIKA1. Many of its modules are common to the quicklook analysis and to lab
characterization. We here give an overview of its structure and of the main
parameters to allow users to have a better intuition of what they actually do
when they change the main parameters without needing to go deep inside the
routines.

\section{Disclaimer}

Except for a few exceptions that we wish to correct when we encounter them, all
the routines of the pipeline are located in {\tt Processing/Pipeline/NIKA\_Pipe}.

\section{The main characters: {\tt param}, {\tt info}, {\tt data}, {\tt kidpar}}

These are the four main input/output of the pipeline routines.

\subsection{param}
{\tt param} is a structure that contains all the input paramters of the
pipeline. They will be recorded in the final map fits header. {\tt param}
is created by {\tt nk\_default\_param.pro} and its parameters can then be changed
by the user. A more complete description of these parameters is given in \ref{se:parameters}.

\subsection{info}
{\tt info} is a structure that gathers all the required information about the
scan, e.g. its scan type (polarized or not, OTF, Lissajou, Pointing.., Ra-Dec or Az-El), source
name, map center coordinates, telescope information... and also information that
are derived during the processing, such as the opacities, or the measured fluxes
and noises on the final maps.

{\tt info} has three special fields:
\begin{itemize}
\item {\tt status}: if it's not 0, it means that an error happened during the
  data processing. if {\tt status = 1}, it is a major crash and the routine
  immediately exists, if it is something else than 1, it is considered as a
  warning only. This allows to process
  many scans at the same time and to be immune at crashes on some problematic
  scans
\item {\tt error\_message}: when an error occurs and an error message is
  produced and stored in this field.
\item {\tt routine}: contains the name of the routine that issued the error message.
\end{itemize}

When the pipeline is launched on a series of scan, if errors occur, they are
logged in an ascii file called {\tt error\_XX.dat} in {\tt param.project\_dir}.


\subsection{data}

{\tt data} is the structure that contains the timelines and their associated
quanties, such as subscan index, time, individual kid pointings, flags, weights for the map making...


\subsection{kidpar}
{\tt kidpar} is the structure that contains all the kid related information that
are not time dependent, such as their pointing offsets, types, fwhm,
calibration... This structure depends on the run. A dedicated routine, {\tt
  nk\_get\_kidpar\_ref.pro} finds the relevant kidpar for a given run and passes
it to the pipeline.\\

So far, the available {\tt kidpar.type} of kids are:
\begin{itemize}
\item 1: this kid is used for the derivation of e.g. common modes, the
  decorrelation and is projected on the final map.
\item 2: this kid is off resonance and is not used for common mode or
  decorrelation (unless explicitely requested by a relevant {\tt
    param.decor\_method}) and is not projected.
\item 3: This kid is a so called ``dark kid'', is not used for common mode or
  decorrelation (unless explicitely requested by a relevant {\tt
    param.decor\_method}) and is not projected.
\item 4 or higher: this kid is either double or has been discarded for some
  other reason. It is not used for common mode or
  decorrelation (unless explicitely requested by a relevant {\tt
    param.decor\_method}) and is not projected.
\end{itemize}


\section{Typical algorithm}

Let's assume you want to produce a map of {\it scan='20150123s128'}, the simpler
way is to type in IDL:\\
{\tt IDL> nk, '20150123s128'}\\

{\tt nk.pro} has initialized {\tt param} and {\tt info} according to {\tt
  nk\_default\_param.pro} and {\tt nk\_default\_info.pro}, processed the data
accordingly, displayed the results and saved them into {\tt
  param.output\_dir/results.save}. If you look into {\tt nk}, you'll see that
all the action is shared between three subroutines: {\tt nk\_scan\_preproc.pro},
{\tt nk\_scan\_reduce.pro} and {\tt nk\_projection\_3.pro}.

\subsection{nk\_scan\_preproc.pro}

This routine reads the raw data from the disk, selects the relevant fraction of
the scan after the tuning at the beginning and before the tunings in the end,
flags and interpolates glitches, flags out speeds with anomalous pointing
coordinates, corrects if possible for pointing inaccuracies, computes kids
individual pointing, calibrates the data.

Everything that happens in this routine does not depend on the decorrelation,
filtering or projection parameters.


\subsection{nk\_scan\_reduce.pro}

This routine takes the calibrated but ``dirty'' data from {\tt
  nk\_scan\_preproc.pro} and cleans them so that they can be projected on a
map.\\

The heart of {\tt nk\_scan\_reduce.pro} is the decorrelation and filtering of
the data in {\tt nk\_clean\_data.pro}. There are two extreme ways to clean the
data, depending on the kind of source  that is observed: point like or diffuse.

\subsubsection{Point sources and decorrelation}

In the case of a point source, or a source that is small compared to the matrix
imprint on the sky, the best way to correct for the atmosphere and correlated
electronic noise is to build a template of all the correlated emissions and
regress out this template from each kid timeline. To build this template without
being biased by the source, we defined a region around it whose radius is given
by {\tt param.decor\_cm\_dmin} (=27.75 arcsec by default). At each time, we
build a common mode template using an average of all the kids that are outside
this region at this time. We then regress each kid timeline against this
template outside the region, and subtract the common mode everywhere, including
in the region.

Depending on the user's choice of parameters, the commmon mode is not computed
exactly in the same way (See sect.~\ref{se:parameters}, but the principle remains.

\subsubsection{Diffuse sources and decorrelation}

If the observed source is large compared to the matrix size, we cannot mask it
and build a common mode like in the previous section because no kid will lie
outside the masked region. The case of diffuse emission is the most challenging
for NIKA(2).

While we are working on the adaptation to the pipeline of performant map making
algorigthms (Scanamorphos and SanePic), we
have designed temporary work arounds. First, the idea is to derive the common
mode everywhere, regress it out of timelines and build a first estimate of the
source. This first estimate can then be backprojected into the timelines,
subtracted, and an improved estimate of the common mode can be derived. This
iterative process should improve the final maps compared to a brute force
projection but has not been qualified completely yet.

In the case of SZ clusters observation, we use the 1mm channel has an atmosphere
monitor and regress it out from the 2mm. This process allows to subtract the
atmosphere on all scales while living most of the large scale SZ emission
intact.

In specific cases, we can merge several approaches, doing a first projection and
then masking its brightest parts before iterating the map making process.


\subsubsection{Remarks on decorrelation and map making}

As for now, there is no standard way to optimally process any kind of
observation. If the user has a priori knowledge on the source, he can taylor the
pipeline parameters to optimize the final maps. In any case, we rely on
simulations to validate the pipeline transfer function.

\subsection{nk\_projection\_3.pro}

This routine handles the projection of the clean data output by {\tt
  nk\_scan\_reduce.pro}. So far, it is a simple inverse variance weighted data
average per pixle, with a Nearest Grid Point in the flat sky approximation.


\section{Main parameters}
\label{se:parameters}

All parameters are defined and briefly documented in {\tt
  nk\_default\_param.pro}. Here is a summary of the main ones with their default
(def) and recommended (rec.) values. The default value are most of the time set for convenience and/or
robustness in lab or quicklook conditions. The recommended values are those that
usually give the best offline data reduction.



\begin{itemize}
\item \underline{\tt flag\_uncorr\_kid = 0 (def) = 1 (rec)} The correlation of all kids to oneanother is
  computed and gives a median kid to kid correlation. If this parameter is set
  to 1, all the kids that are on average less correlated to the other ones are
  suspected to show anomalous behaviour or extra noise and will be discarded
  from the final decorrelation and map.

\item \underline{\tt flag\_sat = 0 (def) = 1 (rec)} If set to 1, all kids whose $I$,$Q$
  angle exceeds {\tt param.flag\_sat\_val=3.d0(def,rec)} is considered as
  saturated and is not used.

\item \underline{\tt flag\_ovlap = 0 (def) = 1 (rec)} if set to 1, by a look at the
  resonnances, most ``double'' kids are discarded from the analysis.

\item \underline{\tt flag\_oor = 0 (def) = 1 (rec)} if set to 1, the pipeline discards kids
  that seem out of resonance.
  
\item \underline{\tt decor\_cm\_dmin = 27.75, (def), (rec)} Minimum distance to
  the source for a sample to be declared "off source" and not used to compute
  the decorrelation common mode at time $t$.

\item \underline{\tt decor\_per\_subscan = 1, (def), (rec)} : if set to 1, a
  common mode is computed and subtraced per subscan rather than on the entire
  scan.

\item \underline{\tt decor\_method = 'COMMON\_MODE', (def)} 'Common\_mode' means
  that the common mode is computed without masking the
  source. 'Common\_mode\_kids\_out' means that the common mode is computed only
  using kids outside the masked region. The masked region is defined in {\tt
    grid.mask} and leads to {\tt data.off\_source = 0} on the masked region, 1
  outside the masked region. ``common\_mode\_one\_block'' means that the
  template is built using only the kids that are most correlated to the current one.

\item \underline{\tt decor\_elevation = 1, (def), (rec)}  Set to 1 to decorrelate
  also from elevation variations

\item \underline{\tt median\_common\_mode\_per\_block = 1, (def), (rec)} Set to 0 to
  compute the median common mode that is used for cross-calibration on all kids. If set to 1, the median common
  mode is computed using only kids from the block. This parameter is active only
  if {\tt param.decor\_method = 'COMMON\_MODE\_ONE\_BLOCK'}.

\item \underline{\tt corr\_block\_per\_subscan = 0, (def), (rec)} Set to 1 to
  recompute the blocks of correlated kids per subscan (provided
  {\tt param.decor\_per\_subscan} is set and {\tt param.decor\_method =
    'COMMON\_MODE\_ONE\_BLOCK'}.

\item \underline{\tt n\_corr\_block\_min = 15, (def), (rec)}  Minimum number of
  kids used to derive the common mode, in order of maximum correlation if {\tt
    param.decor\_method = 'COMMON\_MODE\_ONE\_BLOCK'}.
\item \underline{\tt nsigma\_corr\_block = 2, (def), (rec)} Once the block or
  correlated kids is defined, any other kid correlated with those of the block
  at less than nsigma\_corr\_bloc is added to the block if {\tt
    param.decor\_method = 'COMMON\_MODE\_ONE\_BLOCK'}.

\item \underline{\tt polynomial = 0, (def), (rec)} Set to some degree /= 0 to
  subtract a polynomial per kid, per subscan or per scan depending on {\tt
    param.decor\_per\_subscan}

\item \underline{\tt interpol\_common\_mode = 1, (def), (rec)} Set to 1 to interpolate potential holes in the derived common mode.

%%          ;;---------- Filtering parameters
\item \underline{\tt line\_filter = 0, (def), 1 (rec)} Set to 1 to detect and
  notch filter noise lines (e.g. from pulse tubes).
\item \underline{\tt line\_filter\_width = 2.264962d0, (def), (rec)}  [Hz], width
  used to detect noise lines as excess over the average power spectrum.
\item \underline{\tt line\_filter\_nsigma = 4.d0, (def), (rec)} Threshold to detect noise lines.
\item \underline{\tt line\_filter\_freq\_start = 1.d0, (def), (rec)} Start to look for noise lines above this freq (Hz).
\item \underline{\tt bandpass = 0, (def), (rec)}  Set to 1 to perform a bandpass
  on the timelines.
\item \underline{\tt freqlow = 0.d0, (def), (rec)} Minimum frequency of the bandpass.
\item \underline{\tt freqhigh = 0.d0, (def), (rec)} Maximum frequency of the bandpass.

%%          ;;---------- Weigth parameters
\item \underline{\tt w8\_per\_subscan = 0, (def), (rec)} Set to 1 to compute
  projection weights for each subscan indepedently. The weights are the inverse
  variance of the TOI computed outside the masked region.
\item \underline{\tt map\_bg\_var\_w8 = 1, (def), (rec)}  Set to 1 to weight the coaddition of scans by the variance computed on the map instead of on TOI's
\item \underline{\tt kill\_noisy\_sections = 0, (def), (rec)} Set to 1 to discard
  the most noisy sections of timelines (use only in the presence of very weak
  sources and with extreme care since this creates non linearity).
\item \underline{\tt kill\_noise\_nsigma = 3.d0, (def), (rec)} Threshold used to discard noisy sections of timelines
\item \underline{\tt nsigma\_jump = 4.d0, (def), (rec)} To flag jumps in the data.

%%          ;;---------- Zero level parameters
\item \underline{\tt set\_zero\_level\_full\_scan = 0, (def), (rec)} Set to 1 to
  define a zero level per timeline on the entire scan. The average of the
  timeline outside the masked region is subtracted.
\item \underline{\tt set\_zero\_level\_per\_subscan = 1, (def), (rec)} Same as
  above, but subscan by subscan.

%%          ;;---------- Map parameters
\item \underline{\tt fine\_pointing = 0, (def), 1 (rec)} Set to something non zero to use actual vs commanded positions.
\item \underline{\tt imbfits\_ptg\_restore =  0, (def), (rec)} (for Run 8 data
  only). Set to 1 to use the pointing from elvin (ie included in the raw nika data), 1 means that we use the antenna imbfits pointing.
\item \underline{\tt naive\_projection = 1, (def), (rec)} Means flat sky
  approximation, nearest grid point.
\item \underline{\tt map\_reso = 4d0, (def), (rec)} ; Resolution of the output
  maps in arcsec.
\item \underline{\tt map\_xsize = 400d0, (def), (rec)} ; total width of the
  output maps in arcsec.
\item \underline{\tt map\_ysize = 400d0, (def), (rec)} ; total height of the
  output maps in arcsec.
\item \underline{\tt map\_proj = 'RADEC', (def), (rec)} Or ``azel'', or ``nasmyth''
\item \underline{\tt map\_center\_ra = !values.d\_nan, (def), (rec)}  ; output map
  center Ra in degrees. If set to NaN, then it is automatically taken from the AntennaIMBfits.
\item \underline{\tt map\_center\_dec = !values.d\_nan, (def), (rec)} ; output map
  center Dec in degrees. If set to NaN, then it is automatically taken from the AntennaIMBfits.
\item \underline{\tt do\_fpc\_correction = 0, (def), (rec)} ; set to 1 to apply
  (az,el) pointing corrections. If these pointing corrections do not exist in
  Processing/PointingCorrections, then nothing is done.

%%          ;;---------- Calibration
\item \underline{\tt do\_opacity\_correction = 1, (def), (rec)} Set to 0 to bypass the opacity estimation and correction.

\item \underline{\tt project\_dir = !nika.plot\_dir, (def), (rec)} Directory that will contain all the preproc and scan results of a given project

\item \underline{\tt version = '1', (def), (rec)} Version of the data processing
  that you might want to change along with parameters or pipeline revisions.

\item \underline{\tt output\_dir = !nika.plot\_dir, (def), (rec)} Where results per scan are saved
\item \underline{\tt up\_dir = !nika.plot\_dir, (def), (rec)}  Directory where we put the Unprocessed Files
\item \underline{\tt preproc\_dir = !nika.plot\_dir, (def), (rec)} Where .save with pre-processed data are.

\item \underline{\tt glitch\_width = 100, (def), (rec)} Number of samples to
  define a window on which outlyers are considered as glitches, flagged out and interpolated.
\item \underline{\tt glitch\_nsigma = 5.d0, (def), (rec)} Threshold to declare an
  outlyer to be a glitch.

\item \underline{\tt delete\_all\_windows\_at\_end = 0, (def), (rec)} Set to 1 to remove all plots windows (useful when nk is launched on many scans)

\item \underline{\tt pointing\_accuracy\_tol = 2.d0, (def), (rec)} Tolerance on the azimuth sine fit and the actual ofs\_az in lissajous mode to discard the start/end slews

\item \underline{\tt speed\_tol = 5.d0, (def), (rec)} Tolerance on the regularity
  of the instantaneous scanning speed (arcsec/s). Samples for which speed
  exceeds the average speed +- this tolerance are not projected.

\item \underline{\tt nsample\_min\_per\_subscan = 50, (def), (rec)} If non zero, any subscan with less samples than this parameter will not be projected.

\item \underline{\tt fourier\_opt\_sample = 0,  (def), (rec)} ; set to 1 to
  optimize the legnth of data for Fourier transforms (highly recommended in case
  of Fourier filtering).

\item \underline{\tt math = "PF" (def), (rec)} ``PF'' or ``RF'' depending on which conversion
  from $I$, $Q$, $dI$, $dQ$ to total power you want to use.

\end{itemize}


%----------------------------------------------------------------------------------------
\begin{thebibliography}{}
\end{thebibliography}

\end{document}
