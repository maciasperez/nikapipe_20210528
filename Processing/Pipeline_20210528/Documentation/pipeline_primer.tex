\documentclass[a4paper,10pt]{article}
\usepackage{epsfig}
\usepackage{latexsym}
\usepackage{graphicx}
\usepackage{amsfonts}
\usepackage{amsmath}
\usepackage{xcolor}

\topmargin=-1cm
\oddsidemargin=-1cm
\evensidemargin=-1cm
\textwidth=17cm
\textheight=25cm
\raggedbottom
\sloppy

\definecolor{Blue}{rgb}{0.,0.,1.}
\definecolor{LightSkyBlue}{rgb}{0.691,0.827,1.}
\definecolor{Red}{rgb}{1.,0.,0.}
\definecolor{Green}{rgb}{0.,1.,0.}
\definecolor{Purple}{rgb}{0.5, 0., 0.5}
\definecolor{Try}{rgb}{0.15,0.,1}
\definecolor{Black}{rgb}{0., 0., 0.}

%% %To get DRAFT accross all pages
%% \usepackage{draftcopy}
%% %To replace ``DRAFT'' by ``ON GOING''
%% \draftcopyName{ON GOING}{150}

\title{A NIKA2 IDL pipeline primer}
\author{NIKA2 collaboration}

\begin{document}
\maketitle
\tableofcontents

\abstract{The abstract.}


\section{The Pipeline Philosophy}
\begin{itemize}
\item All the codes necessary to run the pipeline are under {\it
  Processing/Pipeline}. Most of them are under {\it Processing/Pipeline/NIKA\_Pipe}.
\item All parameters that condition the data reduction are gathered in the {\tt
  param} structure, that is defined in {\tt nk\_default\_param.pro} and
  initialized with default values.
\item All the information gathered during the processing that needs to be passed
  to other subroutines and that may be passed as output are gathered in the {\tt
    info} structure.
\item maps are gathered in the {\tt grid} structure, that is either passed in
  input of the pipeline or initialized according to {\tt param} in {\tt
    nk\_init\_grid.pro}.
\item All the TOI's and associated flags, pointings etc.. are gathered in the
  {\tt data} structure.
\end{itemize}

\newpage
\section{nk.pro}

The main pipeline routine is {\tt nk.pro}. It can be called as simply as:
\begin{equation}
  IDL> nk, scanID
\end{equation}

\noindent where scanID can be for example '20150213s264'\footnote{Single quotes are important, IDL
does not accept double quotes for strings that start by a number.}. If you want
to tailor the input parameters, then you'll have to edit a script and do
typically:

\begin{center}
\begin{minipage}{10cm}
{\it
\noindent nk\_default\_param, param ; init the param structure\\
param.map\_xsize = ... ; fill it with your favorite parameters\\
param.map\_ysize = ...\\
param.decor\_method = ...\\
...\\
nk, scanID, param=param ; call nk\\
}
\end{minipage}
\end{center}

Quick overview of {\tt nk}:
\input{nk}

\section{nk\_scan\_preproc.pro}

This routines performs most of the operations that do not depend on
decorrelation and filtering. It also computes the pointing of each individual
kid ({\tt data.dra, data.ddec}) and the associated pixel address for the future
projection ({\tt data.ipix}).

Quick overview of {\tt nk\_scan\_preproc.pro}:

\input{nk_scan_preproc}

\section{nk\_scan\_reduce.pro}

This routine performs decorrelation, filtering, computes the weights that will
be affected to the samples for the projection.

Quick overview of {\tt nk\_scan\_reduce.pro}:
\input{nk_scan_reduce}

\section{nk\_clean\_data\_3.pro}
This routine performs all the decorrelation and filtering requested in {\tt
  param}.
Quick overview of {\tt nk\_clean\_data\_3.pro}:
%\input{nk_clean_data_3}

\section{nk\_projection\_4.pro}
%\input{nk_projection_4}







%% \begin{figure}
%% \begin{center}
%% \includegraphics[clip, angle=0, scale = 0.5]{beam_conventions.eps}
%% \caption{
%% Elliptical beam parameters definition. $\beta$ is the angle between
%% the South/North axis and the $x$ axis of the focal plane. The beam is
%% tilted w.r.t $x$ by an angle $\vartheta$. The detector is sensitive to a
%% direction of polarization $\psi$. The gaussian FWHM in the directions
%% $x$ and $y$ are resp. $2.35\sigma_x$ and $2.35\sigma_y$.}
%% \label{fig:beam_params}
%% \end{center}
%% \end{figure}

%----------------------------------------------------------------------------------------
\begin{thebibliography}{}
\end{thebibliography}

\end{document}
